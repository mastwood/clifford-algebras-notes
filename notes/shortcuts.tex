\stackMath
\newcommand\bighat[1]{%
\savestack{\tmpbox}{\stretchto{%
  \scaleto{%
    \scalerel*[\widthof{\ensuremath{#1}}]{\kern-.6pt\bigwedge\kern-.6pt}%
    {\rule[-\textheight/2]{1ex}{\textheight}}%WIDTH-LIMITED BIG WEDGE
  }{\textheight}% 
}{0.5ex}}%
\stackon[1pt]{#1}{\tmpbox}%
}

\stackMath
\newcommand\bigcheck[1]{%
\savestack{\tmpbox}{\stretchto{%
  \scaleto{%
    \scalerel*[\widthof{\ensuremath{#1}}]{\kern-.6pt\bigvee\kern-.6pt}%
    {\rule[-\textheight/2]{1ex}{\textheight}}%WIDTH-LIMITED BIG WEDGE
  }{\textheight}% 
}{0.5ex}}%
\stackon[1pt]{#1}{\tmpbox}%
}

\stackMath
\newcommand\bigtilde[1]{%
\savestack{\tmpbox}{\stretchto{%
  \scaleto{%
    \scalerel*[\widthof{\ensuremath{#1}}]{\kern-.6pt\sim\kern-.6pt}%
    {\rule[-\textheight/2]{1ex}{\textheight}}%WIDTH-LIMITED BIG WEDGE
  }{\textheight}% 
}{0.5ex}}%
\stackon[1pt]{#1}{\tmpbox}%
}

% \newcommand\bigtilde[1]{\ThisStyle{%
%   \setbox0=\hbox{$\SavedStyle#1$}%
%   \stackengine{-.1\LMpt}{$\SavedStyle#1$}{%
%     \stretchto{\scaleto{\SavedStyle\mkern.2mu\sim}{.5467\wd0}}{.7\ht0}%
% %    .2mu is the kern imbalance when clipping white space
% %    .5467++++ is \ht/[kerned \wd] aspect ratio for \sim glyph
%   }{O}{c}{F}{T}{S}%
% }}

\newcommand{\norm}[1]{\left\lVert#1\right\rVert}

\renewcommand\maketitlehooka{\null\mbox{}\vfill}
\renewcommand\maketitlehookd{\vfill\null}
\addbibresource{bib.bib}
\numberwithin{equation}{section}
\DeclareMathOperator{\Span}{\textsf{span}}
\DeclarePairedDelimiter\ceil{\lceil}{\rceil}
\DeclarePairedDelimiter\floor{\lfloor}{\rfloor}  
\DeclarePairedDelimiter{\innprod}{\langle}{\rangle}
\DeclareMathOperator{\QQ}{\mathbb{Q}}
\DeclareMathOperator{\PP}{\mathbb{P}}
\DeclareMathOperator{\Aa}{\mathcal{A}}
\DeclareMathOperator{\Nn}{\mathcal{N}}
\DeclareMathOperator{\Radical}{\textsf{rad}}
\DeclareMathOperator{\Aut}{\textsf{Aut}}
\DeclareMathOperator{\Mm}{\mathcal{M}}
\DeclareMathOperator{\Bb}{\mathcal{B}}
\DeclareMathOperator{\End}{\textsf{End}}
\DeclareMathOperator{\Ii}{\mathcal{I}}
\DeclareMathOperator{\im}{\textsf{im}}
\DeclareMathOperator{\Cc}{{\mathcal{C}}}
\DeclareMathOperator{\Pp}{\mathcal{P}}
\DeclareMathOperator{\Dd}{\mathcal{D}}
\DeclareMathOperator{\Ff}{\mathcal{F}}
\DeclareMathOperator{\Hh}{\mathcal{H}}
\DeclareMathOperator{\Ss}{\mathcal{S}}
\DeclareMathOperator{\Tt}{\mathcal{T}}
\DeclareMathOperator{\Ll}{\mathcal{L}}
\DeclareMathOperator{\GL}{\textsf{GL}}
\DeclareMathOperator{\RR}{\mathbb{R}}
\DeclareMathOperator{\SL}{\textsf{SL}}
\DeclareMathOperator{\II}{\mathbb{I}}
\DeclareMathOperator{\FF}{\mathbb{F}}
\DeclareMathOperator{\CC}{\mathbb{C}}
\DeclareMathOperator{\ZZ}{\mathbb{Z}}
\DeclareMathOperator{\HH}{\mathbb{H}}
\DeclareMathOperator{\NN}{\mathbb{N}}
\DeclareMathOperator{\diag}{\textsf{diag}}
\DeclareMathOperator{\sgn}{\textsf{sgn}}
\DeclareMathOperator{\nul}{\textsf{null}}
\DeclareMathOperator{\cent}{\textsf{cent}}
\DeclareMathOperator{\twcent}{\textsf{twcent}}
\DeclareMathOperator{\id}{\textsf{Id}}
\DeclareMathOperator{\One}{\mathbbm{1}}
\DeclareMathOperator{\adj}{\textsf{adj}}
\DeclareMathOperator{\p}{\partial}
\DeclareMathOperator{\ksi}{\xi}
\DeclareMathOperator\arctanh{arctanh}
\DeclareMathOperator{\Sym}{\textsf{Sym}}
\DeclareMathOperator{\xx}{{\mathfrak{X}}}
\DeclareMathOperator{\yy}{{\mathfrak{Y}}}
\DeclareMathOperator{\zz}{{\mathfrak{Z}}}
\DeclareMathOperator{\KK}{\mathbb{K}}
\DeclareMathOperator{\signature}{\textsf{signature}}
\DeclareMathOperator{\Var}{\textsf{Var}}
\DeclareMathOperator{\arsinh}{\textsf{arsinh}}
\DeclareMathOperator{\Cov}{\textsf{Cov}}
\DeclareMathOperator{\SU}{\textsf{SU}}
\DeclareMathOperator{\Spin}{\textsf{Spin}}
\DeclareMathOperator{\Pf}{\textsf{Pf}}
\DeclareMathOperator{\image}{\textsf{image}}
\DeclareMathOperator{\Alt}{\textsf{Alt}}
\DeclareMathOperator{\Rank}{\textsf{rank}}
\DeclareMathOperator{\Cl}{C\ell}
\DeclareMathOperator{\rank}{\textsf{rank}}
\DeclareMathOperator{\LL}{\mathbb{L}}
\DeclareMathOperator{\Pin}{\textsf{Pin}}
\DeclareMathOperator{\Ad}{\textsf{Ad}}
\DeclareMathOperator{\Orth}{\textsf{O}}
\DeclareMathOperator{\Unitary}{\textsf{U}}
\DeclareMathOperator{\OO}{\mathbb{O}}
\makeatletter
\newcommand{\uset}[3][0ex]{%
  \mathrel{\mathop{#3}\limits_{
    \vbox to#1{\kern-7\ex@
    \hbox{$\scriptstyle#2$}\vss}}}}
\makeatother
\newcommand{\orthoplus}{\uset{\perp}{\oplus}}
\newcommand{\bigorthoplus}{\smash[b]{\uset{\perp}{\bigoplus}}}
\DeclareMathOperator{\Sp}{\textsf{Sp}}
\renewcommand{\Im}{\textsf{Im}}
\renewcommand{\Re}{\textsf{Re}}
\renewcommand{\mod}{\textsf{ mod }}
\newcommand\m[1]{\begin{bmatrix}#1\end{bmatrix}} 
\newcommand\mdv[2]{\frac{D#1}{D#2}}
\DeclareMathOperator{\SO}{\textsf{SO}}
\makeatletter
\renewcommand{\vec}[1]{\mathbf{#1}}
\def\bign#1{\mathclose{\hbox{$\left#1\vbox to8.5\p@{}\right.\n@space$}}\mathopen{}}
\newcommand{\hk}{\mathbin{\! \hbox{\vrule height0.3pt width5pt depth 0.2pt \vrule height5pt width0.4pt depth 0.2pt}}}
\makeatother

\tcolorboxenvironment{proof}{
  colback=brown!10!white,
  boxrule=0pt,
  boxsep=1pt,
  breakable,
  left=10pt,right=10pt,top=10pt,bottom=10pt,
  oversize=2pt,
  before skip=2\topsep,
  after skip=2\topsep,
}
\theoremstyle{definition}
\newtheorem{example}{Example}
\tcolorboxenvironment{example}{
  colback=blue!4!white,
  boxrule=0pt,
  boxsep=1pt,
  breakable,
  left=10pt,right=10pt,top=10pt,bottom=10pt,
  oversize=2pt,
  before skip=2\topsep,
  after skip=2\topsep,
}
\newtheorem{defn}{Definition}
\tcolorboxenvironment{defn}{
  colback=green!7!white,
  boxrule=0pt,
  boxsep=0pt,
  breakable,
  left=10pt,right=10pt,top=10pt,bottom=10pt,
  oversize=2pt,
  before skip=2\topsep,
  after skip=2\topsep,
}
\newtheorem{lemma}{Lemma}
\tcolorboxenvironment{lemma}{
  colback=yellow!10!white,
  boxrule=0pt,
  boxsep=1pt,
  breakable,
  left=10pt,right=10pt,top=10pt,bottom=10pt,
  oversize=2pt,
  before skip=2\topsep,
  after skip=2\topsep,
}
\newtheorem{thm}{Theorem}
\tcolorboxenvironment{thm}{
  colback=purple!10!white,
  boxrule=0pt,
  boxsep=0pt,
  breakable,
  left=10pt,right=10pt,top=10pt,bottom=10pt,
  oversize=2pt,
  before skip=2\topsep,
  after skip=2\topsep,
}
\newtheorem{cor}{Corollary}
\tcolorboxenvironment{cor}{
  colback=yellow!10!white,
  boxrule=0pt,
  boxsep=0pt,
  breakable,
  left=10pt,right=10pt,top=10pt,bottom=10pt,
  oversize=2pt,
  before skip=2\topsep,
  after skip=2\topsep,
}
\newtheorem*{remark*}{Remark}
\newtheorem*{physics*}{Physics Remark}
\tcolorboxenvironment{physics*}{
  colback=violet!10!white,
  boxrule=0pt,
  boxsep=0pt,
  breakable,
  left=10pt,right=10pt,top=10pt,bottom=10pt,
  oversize=2pt,
  before skip=2\topsep,
  after skip=2\topsep,
}
\numberwithin{thm}{section}
\numberwithin{example}{section}
\numberwithin{defn}{section}
\numberwithin{lemma}{section}
\numberwithin{cor}{section}